\documentclass[12pt] {article}
\usepackage[utf8] {inputenc}
\usepackage {amsmath}
\usepackage {amssymb}
\usepackage {amsthm}
\usepackage {mathtools}

\DeclarePairedDelimiter\floor{\lfloor}{\rfloor}

\theoremstyle{plain}
\newtheorem{thm}{Theorem}[section]
\newtheorem{lemma}[thm]{Lemma}
\newtheorem{prop}[thm]{Proposition}

\theoremstyle{definition}
\newtheorem{defn}[thm]{Definition}
\newtheorem{exmp}[thm]{Example}
\newtheorem{prob}[thm]{Problem}
\newtheorem{algo}[thm]{Algorithm}


\begin {document}

\title {Integer Factorization}
\author {Miska Kananen}
\date {\today}
\maketitle

\tableofcontents

\section {Introduction}

We will define the problem of integer factorization and present some applications for efficient factorization. Then we will look at some trivial and some more efficient factorization algorithms, namely Pollard's Rho algorithm and the linear sieve, and implement them in practice.

\subsection {Definitions}

\begin{defn} (Integer Factorization problem)
Let $n$ be a positive integer. Find an integer $a$ $(1 < a < n)$ such that $n = ab$ for some positive integer $b$ or report that such integer does not exist. We call $a$ a \textit{factor} of $n$ and the product $ab$ a \textit{factorization} of $n$.
\end{defn}

If there exists a factorization of $n$, $n$ is \textit{composite}. Otherwise $n$ is \textit{prime} (or equals $1$, in case of which it is neither). Note that it is not required for $a$ to be prime. 

\begin{exmp} (Factorization)
Let $n = 36$. We can write $n = 4 \cdot 9$. Here $4$ is a factor of $36$ and the product $4 \cdot 9$ is a factorization of $36$.
\end{exmp}

We are interested in solving two specific problems: finding a factor of an integer $n$ and finding a factor for all integers in range $\{1, 2, \dots, n\}$. Let's define these problems more formally.

\begin{prob}
\textit{factor(n)}: find a factor of integer $n$ and return it. If such factor does not exist, report it.
\end{prob}

\begin{prob}
\textit{factor-range(n)}: for each integer $i$ in range $\{1, 2, \dots, n\}$, find a factor of $i$ and return it, or report that such factor does not exist.
\end{prob}

In practice we will handle the case where a factor does not exist by returning $-1$.

\begin{exmp}
\textit{factor(36)} = $4$, \textit{factor(5)} = $-1$
\end{exmp}

\begin{exmp}
\textit{factor-range(10)} = $[-1, -1, -1, 2, -1, 2, -1, 2, 3, 2]$
\end{exmp}

\subsection {Applications}

\section {Basic algorithms}

In this section we look at some easy, but slightly inefficient algorithms for solving \textit{factor} and \textit{factor-range}. In later chapters we will find out how to solve the problems more efficiently.

\subsection {Trial division}

The most trivial way to solve \textit{factor(n)} is to iterate all integers in range $\{2, 3, \dots, (n-1)\}$ and try whether one of them divides $n$. In this case, we have found a factor of $n$. This kind of algorithm is called \textit{trial division}. 

The correctness of the algorithm is easy to see: every possible factor of $n$ is tried, and therefore if one exists, it will be found. The number of divisions done by the algorithm is in the worst case linear to the size of $n$, and the time complexity of the algorithm is thus $O(n)$. However, we can do better.

\begin{prop}
\label {prop:factupperbound}
If $n$ is composite, one of its factors is smaller than or equal to $\sqrt{n}$.
\end{prop}

\begin{proof}
Since $n$ is composite, it can be written as $n = ab$. Assume that $a > \sqrt{n}$ and $b > \sqrt{n}$. Now $ab > \sqrt{n}^2 \implies ab > n$, which is a contradiction. Therefore either $a$ or $b$ must be $\leq \sqrt{n}$.
\end{proof}

According to Proposition \ref{prop:factupperbound}, it suffices to try out the integers from $2$ to $\sqrt{n}$ inclusive. If at this point we have not found a divisor, there are none. Now we do only $\sqrt{n}$ steps in the worst case, and the time complexity of the improved algorithm is $O(\sqrt{n})$. There is still room for improvement.

\begin{prop}
\label {prop:primes6pm1}
All primes $p$ except $2$ and $3$ are of the form $6k \pm 1$ for some integer $k$.
\end{prop}

\begin{proof}
All integers can be represented as $6k + m$ for some integers $k$ and $0 \leq m < 6$. $6k + 0$ is divisible by 6. $6k + 2$ and $6k + 4$ are divisible by 2. $6k + 3$ is divisible by 3. All primes are therefore of the form $6k + 1$ or $6k + 5 \equiv 6k - 1 \mod 6$.
\end{proof}

We can use Proposition \ref{prop:primes6pm1} to reduce the amount of required divisions by only trying $2$, $3$ and the numbers of the form $6k \pm 1$. Here is our final trial division algorithm:
\begin{algo} (Trial division)
\begin{enumerate}
\item Check if $2$ or $3$ divide $n$.
\item Iterate all integers of the form $6k \pm 1$ up to and including $\sqrt{n}$ and check if one of them divides $n$.
\item If no divisor was found, report failure.
\end{enumerate}
\end{algo}

The time complexity of this algorithm is still $O(\sqrt{n})$, but it has a smaller constant factor than the previous one due to a reduced number of divisions.

\subsection {Sieve of Eratosthenes}

\section {Pollard's Rho algorithm}

\section {Linear sieve}

\section {References}

\end {document}
