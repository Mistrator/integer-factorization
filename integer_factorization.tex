\documentclass[12pt] {article}
\usepackage[utf8] {inputenc}
\usepackage[english] {babel}
\usepackage {amsmath}
\usepackage {amssymb}
\usepackage {amsthm}
\usepackage {mathtools}
\usepackage {listings}
\usepackage {color}
\usepackage[backend=biber] {biblatex}

\addbibresource{bibliography.bib}

\lstset{basicstyle=\footnotesize, keywordstyle=\color{blue}, commentstyle=\color{green}, breaklines=true, numbers=left, frame=single}

\DeclarePairedDelimiter\floor{\lfloor}{\rfloor}

\theoremstyle{plain}
\newtheorem{thm}{Theorem}[section]
\newtheorem{lemma}[thm]{Lemma}
\newtheorem{prop}[thm]{Proposition}
\newtheorem{cor}[thm]{Corollary}

\theoremstyle{definition}
\newtheorem{defn}[thm]{Definition}
\newtheorem{exmp}[thm]{Example}
\newtheorem{prob}[thm]{Problem}
\newtheorem{algo}[thm]{Algorithm}


\begin {document}

\title {Integer Factorization}
\author {Miska Kananen}
\date {\today}
\maketitle

\tableofcontents

\section {Introduction}

We will define the problem of integer factorization and present some applications of efficient factorization. Then we will look at some basic and some more efficient factorization algorithms, namely Pollard's Rho algorithm and the linear sieve, and implement them in practice.

\subsection {Definitions}

\begin{defn} (Integer Factorization problem)
Let $n$ be a positive integer. Find an integer $a$ $(1 < a < n)$ such that $n = ab$ for some positive integer $b$ or report that such integer does not exist. We call $a$ a \textit{factor} of $n$ and the product $ab$ a \textit{factorization} of $n$.
\end{defn}

If there exists a factorization of $n$, $n$ is \textit{composite}. Otherwise $n$ is \textit{prime} (or equals $1$, in case of which it is neither). Note that it is not required that $a$ is prime.

\begin{exmp} (Factorization)
Let $n = 36$. We can write $n = 4 \cdot 9$. Here $4$ is a factor of $36$ and the product $4 \cdot 9$ is a factorization of $36$.
\end{exmp}

We are interested in solving two specific problems: finding a factor of an integer $n$ and finding a factor for all integers in range $\{1, 2, \dots, n\}$. Let's define these problems more formally.

\begin{prob}
\textit{factor(n)}: find a factor of integer $n$ and return it. If such factor does not exist, report it.
\end{prob}

\begin{prob}
\textit{factor-range(n)}: for each integer $i$ in range $\{1, 2, \dots, n\}$, find a factor of $i$ and return it, or report that such factor does not exist.
\end{prob}

In practice we will handle the case where a factor does not exist by returning $-1$.

\begin{exmp}
\textit{factor(36)} = $4$, \textit{factor(5)} = $-1$
\end{exmp}

\begin{exmp}
\textit{factor-range(10)} = $[-1, -1, -1, 2, -1, 2, -1, 2, 3, 2]$
\end{exmp}

\subsection {Applications}

\section {Basic algorithms}

In this section we look at some easy, but slightly inefficient algorithms for solving \textit{factor} and \textit{factor-range}. In later chapters we will find out how to solve the problems more efficiently.

\subsection {Trial division}

The easiest way to solve \textit{factor(n)} is to iterate all integers in range $\{2, 3, \dots, (n-1)\}$ and try whether one of them divides $n$. In this case, we have found a factor of $n$. This kind of algorithm is called \textit{trial division}. 

The correctness of the algorithm is easy to see: every possible factor of $n$ is tried, and therefore if one exists, it will be found. The number of divisions done by the algorithm is in the worst case linear to the size of $n$, and the time complexity of the algorithm is thus $O(n)$. However, we can do better.

\begin{prop}
\label {prop:factupperbound}
If $n$ is composite, one of its factors is smaller than or equal to $\sqrt{n}$.
\end{prop}

\begin{proof}
Since $n$ is composite, it can be written as $n = ab$. Assume that $a > \sqrt{n}$ and $b > \sqrt{n}$. Now $ab > \sqrt{n}^2 \implies ab > n$, which is a contradiction. Therefore either $a$ or $b$ must be $\leq \sqrt{n}$.
\end{proof}

According to Proposition \ref{prop:factupperbound}, it suffices to try out the integers from $2$ to $\sqrt{n}$ inclusive. If at this point we have not found a divisor, there are none. Now we do only $\sqrt{n}$ steps in the worst case, and the time complexity of the improved algorithm is $O(\sqrt{n})$. There is still room for improvement.

\begin{prop}
\label {prop:primes6pm1}
All primes $p$ except $2$ and $3$ are of the form $6k \pm 1$ for some integer $k$.
\end{prop}

\begin{proof}
All integers can be represented as $6k + m$ for some integers $k$ and $0 \leq m < 6$. $6k + 0$ is divisible by 6. $6k + 2$ and $6k + 4$ are divisible by 2. $6k + 3$ is divisible by 3. Thus, all primes must be of the form $6k + 1$ or $6k + 5 \equiv 6k - 1 \mod 6$.
\end{proof}

We can use Proposition \ref{prop:primes6pm1} to reduce the amount of required divisions by only trying $2$, $3$ and the numbers of the form $6k \pm 1$. Here is our final trial division algorithm:
\begin{algo} (Trial division)
\begin{enumerate}
\item Check if $2$ or $3$ divide $n$.
\item Iterate all integers of the form $6k \pm 1$ up to and including $\sqrt{n}$ and check if one of them divides $n$.
\item If no divisor was found in steps 1 and 2, report that there are none.
\end{enumerate}
\end{algo}

The time complexity of this algorithm is still $O(\sqrt{n})$, but it has a smaller constant factor than the previous one due to a reduced number of divisions.

\subsection {Sieve of Eratosthenes}

\section {Pollard's Rho algorithm}

Pollard's Rho algorithm is a randomized Monte Carlo algorithm that solves \textit{factor(n)} in $O(\sqrt[4]{n})$ average time and in $O(1)$ space. It was originally invented by John Pollard in 1975\cite{pollard}.

\subsection {Theory}

Assume $n$ is a composite integer and $n = ab$ for some integers $a, b$ $(1 < a, b < n)$. Without loss of generality, assume $a \leq b$. What kind of a randomized approach we could use to factorize $n$?

The first approach could be to randomly generate a sequence of integers $x_1, x_2, \dots, x_k$ from range $2 \dots (n-1)$ and try if one of them divides $n$. There are $n-2$ integers in the range and if $n$ is a product of two primes, the probability of finding a divisor is $\frac{2}{n-2}$. Therefore it takes roughly $n$ attempts to find a divisor this way.

We can do better. Instead of trying to find an integer $x_i$ that divides $n$, we can try to find an integer for which $\gcd(x_i, n) > 1$. In this case the greatest common divisor is a non-trivial factor of $n$. $\gcd(x_i, n) > 1$ for all multiples $a, 2a, \dots, (b-1)a, b, 2p, \dots, (a-1)b$ and there are $a + b - 2$ such multiples. The probability of finding such an integer is $\frac{a+b-2}{n} \approx \frac{\sqrt{n}}{n} = \frac{1}{\sqrt{n}}$ in the worst case when all factors are approximately equal, and thus it takes roughly $\sqrt{n}$ attempts to find a divisor.

Now we are on par with the trial division. To improve the algorithm further, we need to explore the concept of \textit{birthday paradox}.

\begin{prop} (Birthday paradox)
\label {prop:birthdayparadox}
When randomly generating integers $x_1, x_2, \dots, x_k$ with a uniform distribution from the range $1 \dots n$, we need to generate approximately $O(\sqrt{n})$ integers for the probability that two of the generated integers are equal to reach $0.5$.
\end{prop}

\begin{proof}
Proved in CLRS\cite{clrs}, chapter 5.4.1.
\end{proof}

We will utilize the birthday paradox as follows. When randomly generating the integers $x_1, x_2, \dots, x_k$, we will check for all pairs $(x_i, x_j)$ if $\gcd(x_i - x_j, n) > 1$. How soon can we expect to find a suitable pair?

When generating the sequence $x_i$, we are also implicitly generating an induced sequence $x'_i = x_i \mod a$. If we manage to generate two integers $(x_i, x_j)$ such that $x'_i \equiv x'_j \mod a$, then $a$ divides $(x'_i-x'_j)$ and it must hold that $\gcd(x_i-x_j, n) > 1$. Thus, the question reduces to finding how soon we can expect that $x'_i \equiv x'_j \mod a$.

We assumed that $a \leq b$, so according to Proposition \ref{prop:factupperbound} $a \leq \sqrt{n}$. So, we are generating random integers $x'_i$ in range $1 \dots \sqrt{n}$, and according to the birthday paradox we will likely find two equal integers with $O(\sqrt{\sqrt{n}}) = O(\sqrt[4]{n})$ generated integers.

However, we have a problem. To check if there is a repetition in a list of $k$ integers, we have to do approximately $k^2$ pairwise comparisons. If we implemented the algorithm this way, the time complexity would be still $O(\sqrt{n})$.

Instead of assuming a perfect random number generator, we will use a \textit{pseudorandom number generator}, which is a function that maps each integer to another seemingly random number, generating a sequence of apparently random integers. For our purposes, the recommended function is a polynomial of the form

\[
	f(x) = (x^2 + c) \mod n
\]

where $c$ is an arbitrary integer constant. The usual choice is $c = 1$. Values $c = 0$ and $c = -2$ are not recommended \cite{pollard}.

Using the function, we can generate the sequence from a starting value $x_1$ as follows: $x_2 = f(x_1), x_3 = f(x_2)$ and in general, $x_{k+1} = f(x_k)$.

Since we are taking the values mod $n$, we will find a repeated value at some point. Also, since each value depends only on the previous one, the sequence will enter a cycle after encountering a repeated value. The same properties hold for the induced sequence too, as proven in CLRS\cite{clrs}, chapter 31.9.

The average number of steps before entering a cycle is equal to the number of steps before the first repetition in the sequence, which we showed earlier to be approximately O($\sqrt{n}$) for the $x_i$-sequence and O($\sqrt[4]{n}$) for the $x'_i$-sequence.

So, if the $x'_i$-sequence enters a cycle, we have found a factor of $n$. However, if the $x_i$-sequence enters a cycle at the same time, the obtained factor is $n$, which is not useful. In this case, we have to change the initial values.

How can we detect if a sequence has entered a cycle when we don't store the earlier values of the sequence? We need \textit{Floyd's algorithm}\cite{cphb}.

\begin{algo} (Floyd's algorithm)
\begin{enumerate}
\item Initialize two pointers, $t$ and $h$, pointing at the first element of the sequence.
\item Advance $t$ by one step and $h$ by two steps as long as they haven't met again.
\item When $t = h$, we are in a cycle.
\end{enumerate}
\end{algo}

In Pollard's Rho algorithm we will use Floyd's algorithm on the $x_i$-sequence to detect whether it has entered a cycle. Now we are finally ready to describe the entire Pollard's Rho algorithm.

\begin{algo} (Pollard's Rho algorithm)

The integer to be factored is $n$ and function used to generate the pseudorandom sequence is $f$.
\begin{enumerate}
\item Choose a starting value $s$ from range $1 \dots (n-1)$ and assign $t = f(s)$, $h = f(f(s))$.
\item While $\gcd(h-t, n) = 1$, repeat $t = f(t)$, $h = f(f(h))$.
\item If at this point $\gcd(h-t, n) = n$, the algorithm failed to find a factor. Start from beginning with a different $s$.
\item Otherwise we found a factor $\gcd(h-t, n)$, return it.
\end{enumerate}
\end{algo}

There are some points that are important to notice.

\begin{itemize}
\item The factor found by the algorithm is not necessarily prime.
\item If $n$ is prime, the algorithm might not terminate.
\item If $n$ is a perfect square ($n = m^2$ for some integer m), the algorithm might not terminate.
\item Depending on the choice of $f$, the algorithm might fail to find certain factors: for example the usual choice $f(x) = x^2 + 1$ fails to find the factor 2\cite{pollardfail}.
\end{itemize}

\subsection {Implementation}

Implementation in Python 3.

\lstinputlisting[language=Python]{src/pollardrho.py}

\section {Linear sieve}

Linear sieve is a deterministic algorithm that solves \textit{factor-range(n)} in $O(n)$ time and space. It was originally invented by Gries and Misra in 1978\cite{gries}.

\subsection {Theory}

What makes the sieve of Eratosthenes slightly inefficient is that many composite numbers are marked as such multiple times. If we can find a way to process every composite number only once, we can get the time complexity down to $O(n)$.

The main idea is that if we can find an unique representation for every composite integer, then we can generate every such representation only once.

\begin{thm}
\label {thm:comprepr}
Every composite integer $a$ has an unique representation $a = px$, where $p$ is the smallest prime factor of $a$ and $x$ is an integer.
\end{thm}

\begin{proof}
Proved in Gries' and Misra's paper\cite{gries}.
\end{proof}

\begin{cor}
\label {cor:pflimit}
$x \geq p$ and $x$ does not have prime factors that are smaller than $p$.
\end{cor}

\begin{proof}
If $x < p$, $x$ is either prime or it must have a prime factor. In both cases the obtained prime is be smaller than $p$. However, $a = px$ and $p$ is the smallest prime factor of $a$. We have a contradiction.

Also, when $x \geq p$ and $x$ has a prime factor smaller than $p$, we have the same contradiction again.
\end{proof}

We will denote the smallest prime factor of an integer $i$ as $sp(i)$. The linear sieve works as follows: instead of iterating all the integer multiples of $i$ up to the limit $n$, we will only iterate the representations $pi$: $2i$, $3i$, \dots , $p_ki$, where $p_k$ is the largest prime that is smaller or equal to $sp(i)$. We can't go any higher or we would violate Corollary \ref{cor:pflimit} and thus the uniqueness of the representation. 

This way we will find a factor for every composite number, and every composite number is processed exactly once.

\begin{proof}
We have to prove that we will find a factor exactly once for every composite $a$.

First we will prove that we will find a factor. Since $a$ is composite, it can be written as $a = px$ according to Theorem \ref{thm:comprepr}. We process the integers in ascending order, and since $x < a$, we have already processed $x$. While processing $x$, we marked all multiples $2x, 3x, \dots, sp(x)x$ as composite, and $px$ must have been one of these multiples, since according to Corollary \ref{cor:pflimit} $p \leq sp(x)$.

We still have to prove that we will find a factor for $a$ only once. Let's assume that we can find a factor multiple times. From some integer $x$ we can mark $a$ as composite only once, since there is at most one positive integer solution $p$ for $a = px$. This means that we must also be able to write $a$ as $a = p'x'$ for some $x' \neq x$. However, according to Theorem \ref{thm:comprepr} the representation $a = px$ is unique, and thus we have a contradiction.
\end{proof}

\begin{algo} (Linear sieve)
\begin{enumerate}
\item Zero-initialize a $(n+1)$-element array \textit{sp} and create an empty list \textit{primes}.
\item Set $sp(1) = 1$.
\item Iterate all integers $i$ from 2 to $n$. For each integer $i$:
\begin{enumerate}
	\item If $sp(i)$ is still 0, $i$ is prime. Add $i$ to \textit{primes} and set $sp(i) = i$.
	\item Iterate all integers $p \leq sp(i)$ in \textit{primes}. For each integer $p$:
\begin{enumerate}
		\item Set $sp(pi) = i$.
\end{enumerate}
\end{enumerate}
\item Return the array \textit{sp}.
\end{enumerate}
\end{algo}

\subsection {Implementation}

Implementation in Python 3.

\lstinputlisting[language=Python]{src/linearsieve.py}

\printbibliography

\end {document}
